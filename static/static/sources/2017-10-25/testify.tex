\documentclass{article}
\usepackage{fontspec,xltxtra,xunicode,hyperref}
\usepackage[nofonts]{ctex}

\usepackage[fleqn]{amsmath}
\usepackage{amssymb}


% 英文字体,macOS
\setmainfont{Helvetica}
\setsansfont{Helvetica}
\setmonofont{Hack}

% 中文字体,macOS
\setCJKmainfont[BoldFont=STHeitiSC-Medium, ItalicFont=STHeitiSC-Light]{SIL-Hei-Med-Jian}
\setCJKsansfont[BoldFont=STHeitiSC-Medium]{SIL-Hei-Med-Jian}
\setCJKmonofont{SIL-Kai-Reg-Jian}

% 从零开始计录章节
\setcounter{section}{-1}

% 开启索引,xelatex限定
\usepackage{imakeidx}
\makeindex

% 信息
\title{用乘法实现除法}
\author{论极语易-Vonfry}
\date{\today}

\begin{document}
\maketitle

 正文
\section{题设}
用乘法实现除法

\section{证明}

\noindent
假设 $ Q = frac{N}{D} $,则:
\begin{align*}
    Z &= 1 - D \\
    Q &= \frac{N}{D} = froc{N(1+Z)}{D(1+Z)} \\
    &= \frac{N(1+Z)}{(1-Z)(1+Z)} \\
    &= \frac{N(1+Z)}{1-Z^2}
\end{align*}
用$K = 1+Z^2$重复这个过程
\begin{align*}
    Q &= \frac{N(1+Z)}{1-Z^2} \cdot \frac{1+Z^2}{1+Z^2} \\
    &= \frac{N(1+Z)(1+Z^2)}{1-Z^4} \\
\end{align*}
重复N次
\begin{align*}
    & Q = \frac{N}{D} = \frac{N(1+Z)(1+Z^2)(1+Z^3)\dots (1+Z^{2n-1})}{1-Z^{2n-1}} \\
    & \because Z < 1 \Rightarrow  \lim_{N\to 0}Z^{2n-1}=0\\
    & \therefore Q = N(1+Z)(1+Z^2)(1+Z^3)\dots (1+Z^{2n-1})
\end{align*}
对于8位精度,只需要$n=3$即可,而$n=5$则有32位精度。

\end{document}
